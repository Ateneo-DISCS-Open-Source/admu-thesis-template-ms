\section{Mask Detection} \label{mask}

Various studies have been conducted on the development of mask detection due to the pandemic. Among the problems presented with this is that existing facial detection and tracking systems struggle to recognize faces due to the occlusion generated by masks \cite{dangeFaceMaskDetection2022, alayaryFaceMaskedUnmasked2022}. Dinh et. al. \cite{dinhMaskedFaceDetection2022} highlights other problems in this area, such as object scales, mask types, and variations in lighting among others. Despite these challenges, many have tried developing unique methods to address this issue.

Xue et. al. \cite{xueIntelligentDetectionRecognition2020} made use of the RETINAFACE algorithm for mask detection as well as determining if it is worn correctly. Han et. al. \cite{hanMaskDetectionMethod2020} based off their detection method on single-shot detectors, with an additional lightweight network for feature extraction. On the other hand, Zhang \cite{zhangRealTimeDeepTransfer2021} made use of multiple networks such as Caffe and VGG19 for facial mask detection and classification, respectively. In line with Zhang's work on the use of neural networks, many studies have been published using various neural network architectures, particularly YOLO.

Liu and Ren \cite{liuApplicationYoloMask2021} discusses the viability of the YOLO network on mask detection especially in a real-world setting. Other studies have confirmed this viability by making use of YOLOv3 \cite{sevillaMaskVisionMachineVisionBased2021}, more modern iterations like Tiny-YOLOv4 \cite{sathyamurthyRealtimeFaceMask2021} and YOLOv5 \cite{youssryAccurateRealTimeFace2022}, as well as comparing YOLOX with MobileNetV2, with the former outperforming the latter \cite{EvaluationYOLOXMobileNetV2}.
  
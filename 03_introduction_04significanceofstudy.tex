\subsection{Significance of the Study}

Many notable neural network models such as those under the YOLO family which are used for research and practical applications can be resource-intensive, especially if much larger amounts of data are being processed, thus limiting a system's time and resources. The development of more efficient models can address the challenges posed and encourages future development. 

While several studies have explored the individual benefits of attention blocks, depthwise separable convolutions, filter pruning, or a combination of two, there is a notable research gap in the literature when it comes to combining all three techniques together. The study aims to provide a more comprehensive understanding of how these three techniques can synergistically enhance the performance of the YOLOv3 model. The proposed model contributes to advancing model efficiency by exploring techniques with the goal of reducing computational demands while striving to maintain comparable levels of accuracy. It extends the understanding of optimization techniques and their potential use or impact on other deep learning models.

The developed variant may offer several practical advantages, including faster inference, and reduced memory usage. The findings from this study have the potential to pave the way for more sophisticated and effective compression techniques in computer vision tasks, benefiting areas such as object detection, surveillance systems, and autonomous vehicles. In the case of pedestrian detection, a model that can provide nearly equal or even better performance than that of existing models whilst requiring lesser resources or those which can utilize them much better, is especially helpful in managing public safety and security.